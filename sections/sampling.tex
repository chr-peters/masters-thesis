\subsection{Sensitivity Sampling}




\begin{lemma}
    Let $Z \in \mathbb{R}^{n \times d}$ be weighted by
    $w \in \mathbb{R}^n_{>0}$ where $w_i \in \{ v_1, ..., v_t \}$ for
    all $i \in [n]$.
    The range space induced by
    \begin{equation*}
        \mathcal{F}_{probit} = \left\{ w_ig(z_i\beta) \ |\ i \in [n] \right\}
    \end{equation*}
    satisfies
    $\Delta(\mathfrak{R}_{\mathcal{F}_{probit}}) \leq t \cdot (d + 1)$.
\end{lemma}
\begin{proof}
    We partition the functions of $\mathcal{F}_{probit}$ into $t$ disjoint
    classes
    \begin{equation*}
        F_j = \{ w_ig(z_i\beta) \in \mathcal{F}_{probit} \
        |\ w_i = v_j \},\quad j \in [t].
    \end{equation*}
    The functions in each of these classes have an equal
    weight, wich means that by lemma~\ref{lemma:vcdim-constant}, each of
    their induced range spaces has a VC-dimension of at most $d+1$.

    For the sake of contradiction, assume that
    $\Delta(\mathfrak{R}_{\mathcal{F}_{probit}}) > t \cdot (d + 1)$ and let
    $G$ be the corresponding set of size $|G| > t \cdot (d + 1)$ that
    is shattered by $\mathcal{R}(\mathcal{F}_{probit})$.
    Since the sets $F_j$ are disjoint, each intersection
    $F_j \cap G$ must be shattered by $\mathcal{R}(F_j)$.
    Further, at least one of the intersections must have at minimum
    $\frac{|G|}{t}$ elements, which means that for at least one $j \in [t]$
    it holds that
    $|F_j \cap G| \geq \frac{|G|}{t} > \frac{t \cdot (d+1)}{t} = d + 1$.
    This is a contradiction to lemma~\ref{lemma:vcdim-constant}, which
    concludes the proof.
\end{proof}
