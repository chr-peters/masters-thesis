\section{Probit Regression}

\paragraph{Situation:}
We have $n$ data points $(x_i, y_i), \ i=1,...,n$ with
$x_i \in \mathbb{R}^d$ and $y \in \{ -1, 1\}$.

\paragraph{Probit Model:}
$y_i$ is a realization of the random variable $Y_i$.
$Y_1, ..., Y_n$ are independent.
The distribution of $Y_i$ is as follows:
\begin{align*}
    P(Y_i = 1 | x_i; \beta)  & = \Phi(x_i^T \beta)                          \\
    P(Y_i = -1 | x_i; \beta) & = 1 - \Phi(x_i^T \beta) = \Phi(-x_i^T \beta)
\end{align*}
where $\beta \in \mathbb{R}^d$.
It follows that
\begin{align*}
    P(Y_i = y_i | x_i; \beta) = \Phi(y_i x_i^T \beta)
\end{align*}

\paragraph{Likelihood:}
The likelihood of a parameter vector $\beta$ is given as follows:
\begin{equation*}
    L(\beta) = \prod_{i=1}^n P(Y_i = y_i | x_i; \beta) = \prod_{i=1}^n \Phi(y_i x_i^T \beta)
\end{equation*}
The negative log-likelihood that we wish to minimize is:
\begin{equation*}
    \mathcal{L}(\beta) = -\sum_{i=1}^n \log \Phi(y_i x_i^T \beta)
\end{equation*}

\paragraph{The weighted case:}
We introduce sample weights $w_i \in \mathbb{R}_{>0}$
comprising a weight vector $w \in \mathbb{R}_{>0}^n$.
Further, let $g(z) = -\log \Phi(-z)$.
The objective function now becomes:
\begin{equation*}
    f_w(\beta) = \sum_{i=1}^n w_i g(-y_i x_i^T \beta)
\end{equation*}
To make the notation easier, we define $z_i = -y_i x_i^T$ and introduce
the matrix $Z \in \mathbb{R}^{n \times d}$ with row vectors $Z_i = z_i$.
This gives us:
\begin{equation*}
    f_w(\beta) = \sum_{i=1}^n w_i g(z_i \beta)
\end{equation*}

\paragraph{Gradient:}
The gradient of the objective function is needed during optimization.
To derive it, we first need the derivative of $g(z)$:
\begin{equation*}
    g'(z) = \frac{d}{dz} - \log \Phi(-z) = \frac{\phi(z)}{\Phi(-z)}
\end{equation*}
Now we can calculate the gradient of the objective function as follows:
\begin{equation*}
    \frac{\partial f_w(\beta)}{\partial \beta} =
    \sum_{i=1}^n w_i \frac{\partial g(z_i \beta)}{\partial \beta} =
    \sum_{i=1}^n w_i z_i g'(z_i \beta)
\end{equation*}

\begin{lemma}
    Let $g(z) = -\log \Phi(-z)$. Then it holds for all $z \geq 0$ that:
    $$
        \frac{1}{2} z^2 \leq g(z)
    $$
\end{lemma}
\begin{proof}
    The following relationship holds for all $z \geq 1$:
    \begin{align*}
        \Phi(-z) & = \frac{1}{\sqrt{2 \pi}} \int_{-\infty}^{-z} \exp{ \left(-\frac{1}{2} x^2 \right)} dx       \\
                 & \leq \frac{1}{\sqrt{2 \pi}} \int_{-\infty}^{-z} -x \exp{ \left(-\frac{1}{2} x^2 \right)} dx \\
                 & = \frac{1}{\sqrt{2 \pi}} \exp{\left( -\frac{1}{2} z^2 \right)}                              \\
                 & \leq \exp{\left( -\frac{1}{2} z^2 \right)}                                                  \\
    \end{align*}
    We therefore have for $z \geq 1$:
    $$
        e^{g(z)} = e^{-\log \Phi(-z)} = \frac{1}{\Phi(-z)} \geq e^{\frac{1}{2} z^2}
    $$
    Since $\exp( \cdot )$ is a monotonically increasing function,
    it follows that $g(z) \geq \frac{1}{2}z^2$ for all $z \geq 1$.

    \noindent{}Let us now turn to the case when $0 \leq z \leq 1$.
    Both $g(z)$ and $\frac{1}{2}z^2$ are monotonically increasing
    and continuous functions for $0 \leq z \leq 1$.
    Together with the fact that $g(0) > \frac{1}{2}$ it follows
    for all $0 \leq z \leq 1$ that
    $$
        g(z) \geq g(0) > \frac{1}{2} = \max_{0 \leq z \leq 1} \frac{1}{2} z^2 \geq \frac{1}{2} z^2
    $$
    which concludes the proof.
\end{proof}

\begin{lemma}
    Let $g(z) = -\log \Phi(-z)$. Then it holds for all $z \geq 2$ that:
    $$
        g(z) \leq z^2
    $$
\end{lemma}
\begin{proof}
    We first show that $\Phi(-z) \geq \frac{1}{\sqrt{2 \pi}} \frac{z}{z^2 + 1} e^{-\frac{1}{2} z^2}$
    for all $z \geq 0$.
    In order to prove this lower bound, we define
    $h(z) = \Phi(-z) - \frac{1}{\sqrt{2 \pi}} \frac{z}{z^2 + 1} e^{-\frac{1}{2} z^2}$
    and show that $h(z)$ is positive for all $z \geq 0$.
    The derivative $h'(z) = -\sqrt{\frac{2}{\pi}} \frac{e^{-\frac{1}{2} z^2}}{(z^2 + 1)^2}$ is
    negative for all $z$, so $h(z)$ is a monotonically decreasing function.
    Also, it clearly holds that $h(0) > 0$ and
    $\lim_{z \rightarrow \infty} h(z) = 0$. It follows that $h(z) \geq 0$
    for all $z > 0$ which proves the lower bound.

    In the next step, we use this result to show that $e^{z^2} \cdot \Phi(-z) \geq 1$
    for all $z \geq 2$:
    \begin{align*}
        e^{z^2} \cdot \Phi(-z) & \geq e^{z^2} \frac{1}{\sqrt{2 \pi}} \frac{z}{z^2 + 1} e^{-\frac{1}{2} z^2}                           \\
                               & = e^{\frac{1}{2} z^2} \frac{1}{\sqrt{2 \pi}} \frac{z}{z^2 + 1}                                       \\
                               & = e^{\frac{1}{2} z^2} \frac{1}{\frac{4}{3}\left( z^2 + 1 \right)} \frac{\frac{4}{3} z}{\sqrt{2 \pi}} \\
                               & \geq \frac{e^{\frac{1}{2} z^2}}{\frac{4}{3}\left( z^2 + 1 \right)}                                   \\
                               & \geq \frac{e^{\frac{1}{2} z^2}}{e^{\frac{1}{2} z^2}}                                                 \\
                               & = 1
    \end{align*}
    From this it follows directly that $\frac{1}{\Phi(-z)} \leq e^{z^2}$
    and thus we have for all $z \geq 2$:
    $$
        e^{g(z)} = e^{-\log \Phi(-z)} = \frac{1}{\Phi(-z)} \leq e^{z^2}
    $$
    Since $\exp{\left( \cdot \right)}$ is monotonically increasing, the claim
    that $g(z) \leq z^2$ for all $z \geq 2$ follows as a direct consequence.

    The ideas for these proofs are based on the work in~\cite{gaussian_bounds}.
\end{proof}