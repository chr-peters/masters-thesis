\subsection{The Sensitivity Framework}

After having imposed some reasonable restrictions on the
datasets under study, it is now time to think about
how an algorithm that selects a coreset
$\mathcal{C} \subseteq \mathcal{D}$ could be constructed.
One of the first ideas that come to mind to solve such a problem
is the process of random sampling.
After all, why don't we just randomly select a subset of
points from $\mathcal{D}$ of the desired size?
Wouldn't that already solve our problem?

The issue with this approach is that it cannot be guaranteed
that such a uniform random sample will yield a coreset,
i.e. a subset of $\mathcal{D}$ that helps us to obtain a
$(1 \pm \epsilon)$-approximation of the original loss function.
As we will later see in the experiments section, uniform sampling
works reasonably well when the data is well behaved, but fails
terribly when there are a few very important datapoints that
it tends to miss.
Intuitively speaking,
if there are lots of points in a dataset that don't influence the
loss function much, but only a few points that have a big impact
on the loss function, uniform sampling fails because it tends
to miss the few very important points.

It turns out, that one way to remedy the downsides of uniform sampling
is to include a measure of importance in the sampling distribution.
Instead of sampling each datapoint with equal probability,
why don't we construct our sampling distribution in such a way,
that the impact of each point on the loss function is taken into
account? This way, more important points would be assigned
a higher probability to be sampled and less important points
would conversely be assigned a lower sampling probability.
This idea forms the basis of the so called
\textit{sensitivity framework}, an algorithmic framework
introduced in~\cite{feldman-langberg-coresets}
(see also \cite{big-data-tiny-data}), that aims to
find coresets by randomly sampling points proportional to
their importance for the loss function.

In the sensitivity framework, the importance of a point
in a dataset can be thought of the maximum proportion of
the loss function that it can take up in the worst case.
To formalize this intuition, the sensitivity framework shifts
the representation of a dataset as a collection of
points towards the representation as a collection of
\textit{functions}, where each function represents the
loss of a point.
To explain what that means, consider the dataset
$\mathcal{D} = \{(x_i, y_i)\}_{i=1}^n$ with scaled model
matrix $Z$, i.e. the rows of $Z$ are given by vectors
$z_i = -(2y_i-1)x_i$ and the loss function
$f(\beta) = \sum_{i=1}^n w_ig(z_i^T\beta)$, where
$w_1, ..., w_n$ are positive weights.
From now on, we will assign to each point in the dataset the
function $g_i(\beta) = g(z_i^T\beta)$, that represents its
individual contribution to the overall loss function.
This way, we can equivalently represent the dataset $\mathcal{D}$
as the set of functions $F = \{g_1, ..., g_n\}$.

Having made this conceptual change of representing a dataset as
a set of functions, we can now use this new representation to
formalize the concept of importance for each point, according
to wich we later want to sample.
As already hinted at, the importance of a point will be the
maximum share of the loss function, that the loss of the specific
point will take up in the worst case.
This worst case importance is also called the \textit{sensitivity}
of a point, and it was first introduced in
\cite{langberg-schulman-sensitivities}.
A formal definition of this concept, which forms
the basis of the sensitivity framework, is given below.

\begin{definition}[\cite{langberg-schulman-sensitivities}]
    \label{def:sensitivity}
    Let $F = \{ g_1, ..., g_n \}$ be a set of functions,
    $g_i: \mathbb{R}^d \rightarrow \mathbb{R}_{\geq 0}, \ i \in [n]$
    and let $w \in \mathbb{R}^n_{>0}$ be a vector of positive weights.
    The sensitivity of $g_i$ for $f_w(\beta) = \sum_{i=1}^n w_i g_i(\beta)$ is defined as
    \begin{equation*}
        \varsigma_i = \sup_{\beta \in \mathbb{R}^d, \ f_w(\beta) > 0} \frac{w_i g_i(\beta)}{f_w(\beta)}.
    \end{equation*}
    The total sensitivity, i.e. the sum of the sensitivities is $\mathfrak{S} = \sum_{i=1}^n \varsigma_i$.
\end{definition}

The true sensitivity $\varsigma_i$ of a function $g_i$ is usually unknown
and its computation can be expensive, because it involves solving the
original optimization problem, which was indicated in~\cite{braverman-feldman-coresets}.
For this reason, we are usually interested to find efficiently computable
upper bounds $s_i \geq \varsigma_i$ for the sensitivities and then
to draw samples proportional to the upper bounds $s_i$.
As we will see, as long as the sum $S = \sum_{i=1}^n s_i$ of the upper
bounds is sufficiently small, the coreset size will be small as well.

The second element of the sensitivity framework, which
\cite{feldman-langberg-coresets} related to the
concept of sensitivity sampling in order to obtain small coresets,
is the theory of range spaces and the VC-dimension.
Its relevant definitions are given below.

\begin{definition}[\cite{feldman-langberg-coresets}]
    A range space is a pair $\mathfrak{R} = (F, \textup{ranges})$, where F is a set
    and $\textup{ranges}$ is a family (set) of subsets of F.
\end{definition}

\begin{definition}[\cite{feldman-langberg-coresets}]
    The VC-dimension $\Delta(\mathfrak{R})$ of a range space
    $\mathfrak{R} = (F, \textup{ranges})$ is
    the size $|G|$ of the largest subset $G \subseteq F$ such that
    \begin{equation*}
        \left| \left\{ G \cap R \ | \ R \in \textup{ranges} \right\} \right|
        = 2^{|G|},
    \end{equation*}
    i.e. $G$ is shattered by $\textup{ranges}$.
\end{definition}

\begin{definition}[\cite{feldman-langberg-coresets}]
    Let $F$ be a finite set of functions mapping from $\mathbb{R}^d$ to $\mathbb{R}^{\geq 0}$.
    For every $\beta \in \mathbb{R}^d$ and $r \geq 0$, let
    \begin{equation*}
        \textup{range}(F, \beta, r) = \left\{ f \in F \ | \  f(\beta) \geq r  \right\}
    \end{equation*}
    and let
    \begin{equation*}
        \textup{ranges}(F) = \left\{ \textup{range}(F, \beta, r) \ | \ \beta \in \mathbb{R}^d, \ r \geq 0  \right\}.
    \end{equation*}
    Then we call $\mathfrak{R}_F := (F, \textup{ranges}(F))$ the range space induced by F.
\end{definition}

The following theorem is the basis of the sensitivity framework and
combines the theory of range spaces with the concept of
sensitivity sampling. Its original version goes back to
\cite{feldman-langberg-coresets}, but it was
further improved by
\cite{braverman-feldman-coresets}.
In this work, we will use the following variant by~\cite{big-data-tiny-data}:

\begin{theorem}[\cite{big-data-tiny-data}]
    \label{theorem:sensitivity-framework}
    Let $F = \{ g_1, ..., g_n \}$ be a set of functions,
    $g_i: \mathbb{R}^d \rightarrow \mathbb{R}_{\geq 0}, \ i \in [n]$
    and let $w \in \mathbb{R}^n_{>0}$ be a vector of positive weights.
    Let $\epsilon, \delta \in (0, \frac{1}{2})$.
    Let $s_i \geq \varsigma_i$ be upper bounds of the sensitivities and
    let $S = \sum_{i=1}^n s_i$.
    Given $s_i$, one can compute in time $O(|F|)$ a set
    $R \subseteq F$ of
    \begin{equation*}
        |R| \in O \left( \frac{S}{\epsilon^2} \left( \Delta \log S + \log \left( \frac{1}{\delta} \right) \right) \right)
    \end{equation*}
    weighted functions, such that with probability $1 - \delta$ we have
    for all $\beta \in \mathbb{R}^d$ simultaneously
    \begin{equation*}
        (1-\epsilon) \sum_{g_i \in F} w_i g_i(\beta) \leq \sum_{g_i \in R} u_i g_i(\beta) \leq (1 + \epsilon) \sum_{g_i \in F} w_i g_i(\beta).
    \end{equation*}
    Each element of $R$ is sampled independently with probability
    $p_j = \frac{s_j}{S}$ from $F$, $u_i = \frac{S w_j}{s_j |R|}$
    denotes the weight of a function $g_i \in R$ that corresponds to
    $g_j \in F$ and $\Delta$ is an upper bound on the
    VC-dimension of the range space $\mathfrak{R}_{F^\ast}$ induced by
    $F^\ast$, where $F^\ast$ is the set of functions $g_i \in F$
    scaled by $\frac{S w_i}{s_i |R|}$, i.e.
    $F^\ast = \left\{ \frac{S w_i}{s_i |R|} g_i(\beta) \ |\ i \in [n] \right\}$.
\end{theorem}

From this theorem, it follows that there are two things that have to be
done in order to find a small coreset for probit regression.

The first one is to find small and efficiently computable upper bounds
on the sensitivities and the second thing is to find a
small upper bound on the VC-dimension of the range space induced by $F^\ast$.
We will do both in the following section.